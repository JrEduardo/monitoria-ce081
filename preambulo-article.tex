%% ======================================================================
%% Eduardo Junior
%% eduardo.jr@ufpr.br
%% 22-11-2015
%% ======================================================================
%% Template latex article para ser usado na opção `include: in_header:`
%% de arquivos Rmd com ouput pdf_document

%% ======================================================================
%% Fontes

\usepackage{palatino}
\usepackage{eulervm}
\usepackage[none]{ubuntu}
\usepackage{verbatim}

\renewcommand{\ttdefault}{ubuntumono}
\renewcommand{\ttfamily}{\fontUbuntuMono}
\makeatletter
\def\verbatim@font{\small\fontUbuntuMono}
\makeatother

%% ======================================================================
%% Colorindo ambiente verbatim

\usepackage{etoolbox}
\makeatletter
\preto{\@verbatim}{\topsep=2pt \partopsep=2pt}
\makeatother

\let\oldv\verbatim
\let\oldendv\endverbatim

\def\verbatim{\setbox0\vbox\bgroup\oldv}
\def\endverbatim{\oldendv\egroup \hspace*{-0.6ex}\colorbox[gray]{0.93}{\usebox0}}
% \def\endverbatim{\oldendv\egroup \usebox0}

%% ======================================================================
%% Define marcações de topo de página
\usepackage{fancyhdr}
\fancyhf{}
\fancyhead[L]{Eduardo E. R. Junior \& Andryas Waurzenczak}
\fancyfoot[R]{\bfseries{\thepage}}
%\fancyfoot[C]{RESTRICTED DISTRIBUTION}
\renewcommand{\headrulewidth}{1pt}
\renewcommand{\footrulewidth}{0pt}

\pagestyle{fancy}

\makeatletter
\let\ps@plain\ps@fancy 
\makeatother

%% ======================================================================
%% Define parâmetros para elaboração de exercícios
\usepackage[final, answers]{probsoln}
  \setkeys{probsoln}{fragile} % Para permitir verbatim no \defproblem.

\usepackage{xcolor}
\definecolor{colres}{HTML}{808080}

\newcommand{\resp}[1] {
\textbf{\footnotesize #1}
}

\newenvironment{myalign*}
 {\setlength{\abovedisplayskip}{0pt}\setlength{\belowdisplayskip}{0pt}%
  \csname flalign*\endcsname}
 {\csname endflalign*\endcsname\ignorespacesafterend}

%% ======================================================================
%% Pacotes
\usepackage{multicol}
\usepackage[bottom]{footmisc}
\usepackage{float}
\usepackage{lipsum}
\usepackage{enumitem}
\usepackage{amsmath}
\usepackage{amsfonts}
\usepackage{amssymb}
\usepackage{mathtools}
\usepackage{multicol}